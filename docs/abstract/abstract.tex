\documentclass[acmsmall,nonacm]{acmart}
\usepackage{iftex}
% \ifPDFTeX
%   \usepackage{libertinust1math}
%   \usepackage{libertine}
% \else
%   \usepackage{libertinus}
% \fi
% \usepackage[scaled=0.8]{FiraMono}
\usepackage{indentfirst}
% \usepackage{listings}
\usepackage[newfloat]{minted}
\usepackage{etoolbox}
\AtBeginEnvironment{minted}{\singlespacing%
    \fontsize{10}{10}\selectfont}
\setminted{baselinestretch=0.7}
\setminted{fontsize=\footnotesize}

\usepackage{caption}

\usepackage{xcolor}

\usepackage{amsmath}
% \usepackage{amssymb}
\usepackage{amsfonts}
\usepackage{amsthm}
\usepackage{thmtools}
\usepackage{mathtools}
\usepackage{mathpartir}
% \usepackage{stmaryrd}
\usepackage{bm}

\usepackage{multicol}

\usepackage{tikz}

\usepackage{booktabs}
\usepackage{enumitem}
\setlist{itemsep=1pt}

% \usepackage{biblatex}

% \usepackage[colorlinks=true, linkcolor=purple]{hyperref}
\usepackage[capitalise, noabbrev]{cleveref}

\usepackage{subfiles}

\numberwithin{figure}{section}
\numberwithin{table}{section}

\makeatletter
\renewcommand\tableofcontents{\@starttoc{toc}}

\declaretheorem[name=Theorem, parent=section]{theorem}
\declaretheorem[name=Definition, parent=section, style=definition]{definition}

% \usepackage[
% backend=biber,
% style=alphabetic,
% sorting=ynt
% ]{biblatex}
% \addbibresource{refs.bib}

%
% Judgments
%
\newcommand{\judgment}[3]{\inferrule{#1}{#2}~~{\textsc{\small [#3]}}}
\newcommand{\judgbox}[1]{\noindent \fbox{$#1$}}

% Contexts
\newcommand{\ctx}{\ensuremath{\Gamma}}
\newcommand{\extendCtx}[3]{\ensuremath{#1 ; ~\assignType{#2}{#3}}}
\DeclareMathOperator{\dom}{dom}
\newcommand{\inCtx}[3]{\ensuremath{#2, #3 \in #1}}
\newcommand{\notInCtx}[2]{\ensuremath{#2 \not\in \dom(#1)}}
\newcommand{\withCtx}[2]{\ensuremath{#1 \vdash #2}}

% Typing
\newcommand{\assignType}[2]{\ensuremath{#1 : #2}}
\newcommand{\ctxAssignType}[3]{\ensuremath{\withCtx{#1}{\assignType{#2}{#3}}}}
\newcommand{\synType}[2]{\ensuremath{#1 \Rightarrow #2}}
\newcommand{\ctxSynType}[3]{\ensuremath{\withCtx{#1}{\synType{#2}{#3}}}}
\newcommand{\anaType}[2]{\ensuremath{#1 \Leftarrow #2}}
\newcommand{\ctxAnaType}[3]{\ensuremath{\withCtx{#1}{\anaType{#2}{#3}}}}

%
% Types
%

% Value types
\newcommand{\TAMName}{\textsf{Value Type}}
\newcommand{\TAMV}{\ensuremath{A}}
\newcommand{\TAInt}{\ensuremath{\textsf{Int}}}
\newcommand{\TABool}{\ensuremath{\textsf{Bool}}}
\newcommand{\TAString}{\ensuremath{\textsf{String}}}
\newcommand{\TAThunk}[1]{\ensuremath{U ~#1}}

% Computation types
\newcommand{\TBMName}{\textsf{Computation Type}}
\newcommand{\TBMV}{\ensuremath{B}}
\newcommand{\TBReturn}[1]{\ensuremath{F ~#1}}
\newcommand{\TBArrow}[2]{\ensuremath{#1 \to #2}}

%
% Values
%
\newcommand{\AMName}{\textsf{Value}}
\newcommand{\AMV}{\ensuremath{a}}

% Literals
\newcommand{\ALit}[1]{\ensuremath{\underline{#1}}}
\newcommand{\AInt}{\ensuremath{\ALit{n}}}
\newcommand{\AString}{\ensuremath{\ALit{s}}}
\newcommand{\ATrue}{\ensuremath{\textsf{tt}}}
\newcommand{\AFalse}{\ensuremath{\textsf{ff}}}

% Thunks
\newcommand{\AThunk}[1]{\ensuremath{\{ #1 \}}}

%
% Computations
%
\newcommand{\BMName}{\textsf{Computation}}
\newcommand{\BMV}{\ensuremath{b}}

% Binding
\newcommand{\BLet}[3]{\ensuremath{\textsf{let}~ #1 = #2 ~\textsf{in}~ #3}}
\newcommand{\BLetAnn}[4]{\ensuremath{\textsf{let}~ #1 : #2 = #3 ~\textsf{in}~ #4}}
\newcommand{\BLetRec}[4]{\ensuremath{\textsf{let rec}~ #1 : #2 = \AThunk{#3} ~\textsf{in}~ #4}}
\newcommand{\BDo}[3]{\ensuremath{\textsf{do}~ #1 \gets #2 ~\textsf{in}~ #3}}
\newcommand{\BDoAnn}[4]{\ensuremath{\textsf{do}~ #1 : #2 \gets #3 ~\textsf{in}~ #4}}

% Lambda
\newcommand{\BLam}[2]{\ensuremath{\lambda #1. ~#2}}
\newcommand{\BLamAnn}[3]{\ensuremath{\lambda #1 : #2. ~#3}}
\newcommand{\BAp}[2]{\ensuremath{#1 ~#2}}

% Force
\newcommand{\BForce}[1]{\ensuremath{! #1}}

% Return
\newcommand{\BReturn}[1]{\ensuremath{\textsf{return}~ #1}}

% Conditional
\newcommand{\BIf}[3]{\ensuremath{\textsf{if}~ #1 ~\textsf{then}~ #2 ~\textsf{else}~ #3}}

% Data & Codata
\newcommand{\DeclName}{\textsf{Declarations}}
\newcommand{\VPName}{\textsf{Var-Param}}
\newcommand{\VPV}{\ensuremath{v^*}}
\newcommand{\MatcherName}{\textsf{Matcher}}
\newcommand{\ComatcherName}{\textsf{Comatcher}}
\newcommand{\MatcherV}{\ensuremath{\varphi_m}}
\newcommand{\ComatcherV}{\ensuremath{\varphi_c}}


\title{Zydeco: A Stack-Based Functional Programming Language}

\begin{document}
\renewcommand{\thesection}{\Alph{section}}

\maketitle

\section{Motivation}

\textit{Call-By-Push-Value(CBPV)}, introduced by Paul Blain Levy, is a programming paradigm that exploits stack for passing arguments when calling functions. By explicitly separating values and computations under the slogan "a value is, a computation does", CBPV has shown great potential for expressing computations. Though motivated by reconciling call-by-name (CBN) and call-by-value (CBV), CBPV ended up developing a powerful stack semantics that is ideal for expressing evaluation order and side effects.

The goal of this project is to experiment with the potential of call-by-push-value. To be specific, we want to encode the status of the stack, i.e. the status of the computation currently being performed, into the type system. By leveraging the stack semantics, we aim to build a functional programming language that is easy for the programmer to reason about the situation of the stack and operate it directly. We claim that codata is a good fit for manipulating the stack by explicitly doing pattern matching on the stack top.

\section{Encoding Calling Conventions In Types}

It would be nice if the programmer can just tell from the type signature whether the computation is going to be eagerly evaluated or lazy evaluated. Consider the type of a regex matcher: Rule -> String -> Bool. As a user of this matcher, we may wonder what's the exact behavior of the matcher: is it supposed to be a ``compiler'' that takes the rule and turns it into a state machine with fast checking speed, or merely a curried version of an ``interpreter'' that inspects the rule on the fly? Staring at the type, there is no hint of how the function is supposed to be used: is it useful to pass in a rule and cache it for later use or not?

\begin{listing}[h]
    \begin{multicols}{3}
        \inputminted{Haskell}{examples/mappre1.zydeco}
        \vfill
        \inputminted{Haskell}{examples/mappre2.zydeco}
        \vfill
    \end{multicols}
    \begin{multicols}{2}
        \inputminted{Haskell}{examples/map1.zydeco}
        \vfill
        \inputminted{Haskell}{examples/map2.zydeco}
        \vfill
    \end{multicols}
    \captionof{listing}{The comparison between an eager and a lazy version of map}
    \label{code: map}
\end{listing}


A type system that adopts call-by-push-value can ease such confusion. All types are classified by two categories: \textit{Value $\mathcal{A}$} and \textit{Computation $\mathcal{B}$}. All computations are done on objects $b$ of sort $\mathcal{B}$, while all objects $a$ of sort $\mathcal{A}$ are delivered as is. To defer or freeze a computation $b$, one can wrap it in a \textit{thunk}, denoted by $\{ b \}$, which is an instance of sort $\mathcal{A}$. To return a value as is, one can use \textit{return}, denoted by $\texttt{ret}~a$. To unwrap a thunk and continue the calculation, one can use the \textit{force} operator, denoted by $!~a$. And finally, to complete computation and bind the returned result to a variable, one can use the \textit{do} structure, denoted by $\texttt{do}~x \leftarrow b_1;~b_2$. The key observation here is that by introducing thunks and ways of manipulating thunks, the programmer gains much better control over computation.

Adding thunks significantly changes our type system by bringing in $U$ types - thunk types, and $F$ types - returner types. Just as the name indicates, a thunk-captured computation will have $U B$ types, while return values will have $F A$ types. The type system is extended with the following rules:

% rules

Listing \ref{code: map}. demonstrates two versions of $\texttt{map}$ functions: the first one demands both $\texttt{f}$ and the list passed in to start computing, while the second one is curried and can return a concrete mapper that can be cached and applied later.

\section{Manipulating Stack With Codata}

\inputminted{Haskell}{examples/abort.zydeco}

\begin{listing}[h]
    \begin{multicols}{2}
        \inputminted{Haskell}{examples/sumfn.zydeco}
        \vfill
        \inputminted{Haskell}{examples/sumco.zydeco}
        \vfill
    \end{multicols}
    \captionof{listing}{Function as codata type}
    \label{code: codata}
\end{listing}

\section{Compilation}

\section{Previous Work on Call-By-Push-Value}

\section{Current Status and Future Goals}

Currently, the project is still at an early stage. By far our implementation covers the parser and a small type checker supporting monomorphic inductive and coinductive data types, and a CEK-machine style evaluator.

We will also explore the possibility of extending the language with more features.

\end{document}
